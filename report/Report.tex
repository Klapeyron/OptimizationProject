\documentclass[11pt]{article}
	\usepackage[T1]{fontenc}
	\usepackage[english]{babel}
	%\usepackage[latin2]{inputenc}
	\usepackage[utf8]{inputenc}
	\usepackage{graphicx}
	\usepackage{float}
	\usepackage{amsfonts, amsmath}
	\usepackage{fancyhdr}
	\usepackage{setspace}
	\usepackage{epstopdf}
	\usepackage{listings}
	\usepackage{color}
	\usepackage{xspace}
	\newcommand{\latex}{\LaTeX\xspace}
	%\graphicspath{images/lab_1}
	
	\pagestyle{fancy}
	\fancyhead{}
	\fancyfoot{}
	\fancyfoot[R]{\thepage}
	
%	\textwidth 16cm
	\textheight 21.0cm
	\topmargin -0.5cm
%	\oddsidemargin 0.5cm
%	\evensidemargin 0.5cm
%	\def\thefootnote{\arabic{footnote})}
\setlength{\parindent}{0ex}
\setlength{\parskip}{1em}

\definecolor{mygreen}{rgb}{0,0.6,0}
\definecolor{mygray}{rgb}{0.5,0.5,0.5}
\definecolor{mymauve}{rgb}{0.58,0,0.82}

\lstset{ %
  backgroundcolor=\color{white},   % choose the background color; you must add \usepackage{color}
  basicstyle=\footnotesize,        % the size of the fonts that are used for the code
  breaklines=false,                 % sets automatic line breaking
  commentstyle=\color{mygreen},    % comment style
  frame=single,	                   % adds a frame around the code
  keepspaces=true,                 % keeps spaces in text, useful for keeping indentation of code (possibly needs columns=flexible)
  language=C++,                 % the language of the code
  numbers=left,                    % where to put the line-numbers; possible values are (none, left, right)
  numbersep=5pt,                   % how far the line-numbers are from the code
  numberstyle=\tiny\color{mygray}, % the style that is used for the line-numbers
  stepnumber=1,                    % the step between two line-numbers. If it's 1, each line will be numbered
  stringstyle=\color{mymauve},     % string literal style
  tabsize=2,	                   	 % sets default tabsize to 2 spaces
  title=\lstname                   % show the filename of files included with \lstinputlisting; also try caption instead of title
}

\renewcommand{\headrulewidth}{0pt}
%\renewcommand{\figurename}{Figure}
			
\begin{document}
\setstretch{1.0}
	%\begin{titlepage}
	\begin{center}
		\begin{tabular}{| c | c |}
		\hline
		Wroclaw University of Technology & Dr Ewa Szlachcic\\ \hline
		Faculty of electronics & Optimization Theory and
		Numerical Methods \\
		Control Engineering and Robotics & Project - AREU0003\\
		$2^{nd}$ cycle studies & \\ \hline
		\multicolumn{2}{c}{Non-linear programming for multicriterial problems}\\
			\hline
	\end{tabular}			
	\end{center}
	
	\textbf{Topic:} Bicriterial optimization of non-linear functions with
	constraints - Niched Pareto Genetic Algorithm (NPGA).
	
	\begin{center}
		\begin{tabular}{| l | c |}
		\hline
		Authors & Tomasz Bartos, 209248\\
				& Radoslaw Zwolski, 209124\\ \hline
		Project group & Monday, 13:15-15:15, Odd weeks\\ \hline
		Project due date & 22.05.2017\\
		\hline
	\end{tabular}
	\end{center}
	
	\newpage

	\section{Introduction}
	
	The aim of the project was to implement an optimization algorithm of non-linear
	bicriterial problem. Niched-Pareto Genetic Algorithm (NPGA) has been used to
	solve this problem. Whole program has been implemented in C++ using QT 	
	framework and allows user to provide functions to optimize as well 
	as configuration parameters by GUI (Graphical User Interface). The result of 
	algorith has been shown as Pareto chart and table of non-dominated points.
	
	\section{Description of the problem}
	
	The algorithm solves a bicriterial problem. Here it minimizes system of two
	functions with M variables.
	
	$$\begin{cases}
		F_{1}(x_{1},x_{2},...,x_{M})\\
		F_{1}(x_{1},x_{2},...,x_{M})
	\end{cases}$$
	
	Constraints constraints given as an input of algorithm are constraints of
	variables. So
	
	$$\begin{cases}
		x_{1min} \leq x_{1} \leq x_{1max}\\
		x_{2min} \leq x_{2} \leq x_{2max}\\
		.\\
		.\\
		.\\
		x_{Mmin} \leq x_{M} \leq x_{Mmax}\\
	\end{cases}$$
	
	First N random points are being generated using uniform distribution as
	$P_{0}$.
	Algorithm iterates over that set of all points and chooses 
	
		
	\section{Estimation of mean and variance}		
	\subsection{Central limit theorem and law of large numbers}
	
	According to central limit theorem (CLT) adding independent random variable
	to an experiment results with their sum tending towards normal distribution.
	It will be shown later.
	
	Moreover here it will be shown that law of large numbers (LLN) is true.
	It says that the more samples there are considered in an experiment closer
	to expected value the result is.
	
	Also what is interesting both theorems (LLN and CLT) are true for any
	distribution of random variable that is considered. 	
	
	\subsection{Estimation of expected value}
	
	Given a set of samples $X={x_{1},\ x_{2},\ ...,\ x_{N}}$ we can estimate
	its expected value calculating:
	
	$$\hat{E}\{X\}=\mu=\frac{1}{N}\sum_{i=1}^{N}x_{i},$$
	
	where N is amount of samples in the set.
	
	During laboratories it has been tried to prove LLN and CLT by an experiment,
	so in order to do it a set of samples has been generated using normal
	distribution and it has been done for many different amount of samples.
	The result can be seen on figures below.
	
%	\begin{figure}[H]
%	\caption{Plot showing value of expected value and how it depends on
%		amount of samples in an experiment}
%	\centering
%	\includegraphics[scale=0.7]{ex_plot_convergence}
%	\label{fig:ex_conv}
%	\end{figure}
	
\end{document}